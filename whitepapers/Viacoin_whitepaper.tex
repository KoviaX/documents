\documentclass{article}

\usepackage{graphicx}
\usepackage{blindtext}
\usepackage[a4paper,width=170mm,top=18mm,bottom=22mm,includeheadfoot]{geometry}
\usepackage[english]{babel} % Language hyphenation and typographical rules
\usepackage{hyperref} % For hyperlinks in the PDF
\usepackage[sc]{mathpazo}
\usepackage{qtree}
\usepackage{forest}
\usepackage{amsmath}
\usepackage{url}
\usepackage[utf8]{inputenc}    % utf8 support       %!!!!!!!!!!!!!!!!!!!!
\usepackage[T1]{fontenc}       % code for pdf file  %!!!!!!!!!!!!!!!!!!!!
\usepackage[backend=biber]{biblatex}
\addbibresource{uni.bib}

\author{Viacoin Dev Team}
\title{Viacoin Whitepaper}
\date{September 12, 2017}


\begin{document}

\maketitle

{\normalfont
      Last updated on \today}

\begin{abstract}\noindent
\normalsize Viacoin is an open source crypto-currency created in 2014,
derived from the \cite{bitcoin2008}Bitcoin protocol that supports embedded consensus with an
extended OP\_RETURN of 120 byte. Viacoin features Scrypt Merged mining,
also called Auxiliary proof of work or AuxPoW, and a 25x faster transactions than Bitcoin.
Viacoin mining reward halving takes place every 6 months and has a total supply of
23,000,000 coins. The inflation rate and mining reward is very low. As the block 
reward of Viacoin is low, miners are given incentive through Merged mining (AuxPoW).
Viacoin is currently mined by one of the biggest mining pools
(F2Pool) with a very high hashrate.
\linebreak
\noindent Other features include a mining difficulty adjustment algorithm to address flaws in Kimoto’s Gravity Well (DarkGravityWave),
Versionbits to allow for 29 simultaneous Soft Fork changes to be implemented at a time, Segwit and the Lightning Network
\end{abstract}


\vfill \noindent
\small Note: The whitepaper, documentation, designs are in research and development phase and subject to change.
\newpage

\normalsize

\section{Scrypt}\label{sec: Scrypt}
In cryptography, \cite{scrypt}Scrypt is a password based key derivation function created by Colin Percival. The algorithm was designed to make it costly to perform large-scale custom hardware attacks by requiring large amounts of memory. In 2012, the algorithm was published by the IETF as an internet draft intended to become an informational RFC, but a version of Scrypt is now used as a proof of work scheme by cryptocurrencies like Viacoin. 
\newline \newline \noindent
Scrypt is a memory hard key derivation function, it requires a reasonably large amount of Random Access Memory to be evaluated.
This makes implementation in special purpose custom hardware (ASICs) require more VLSI area, which would make it unprofitable to build for the purpose of mining Viacoins. The requirement of Scrypt algorithm is a large array of pseudo random bits to be held in memory and a key that is derived from this. The algorithm is based on TMTO (Time-Memory Tradeoff). ASIC advantage in Viacoin is reduced by a factor of 10 compared to Bitcoin. 
\newline \newline \noindent
Scrypt uses the following parameters to generate a derived key:
\begin{itemize}
\item Passphrase: String of characters to hash
\item Salt: Random string provided to Scrypt functions
\item N: Memory/CPU cost parameter
\item P: Parallelization parameter
\item R: Blocksize parameter
\item dkLen: Intended length of the key derived key in bytes
\end{itemize}
$kd = scrypt(P, S, N, P, R, dkLen)$
\newline \newline \noindent
Viacoin parameters where N=1024, R=1, P=1 and S= random 80 bytes producing a 256-bit output

\section{Merged Mining AuxPoW}\label{sec:Merged Mining AuxPoW}

Viacoin \cite{auxpow}Merged mining aims to reuse the mining power of any other \cite{scrypt}Scrypt coin
to add security to the Viacoin blockchain,
allowing a miner to mine for more than one blockchain at the same time. For example, a miner
could mine Viacoin and Litecoin or any other Scrypt coin together with Viacoin with little to no impact on hashrate on either one.
\newline \newline \noindent
Every hash the miner contributes is for the total hashrate of both
cryptocurrencies and results in a more secure blockchain.
An AuxPoW block is a type of block similar to a standard Bitcoin block with two
differences. The hash of the block header does not meet the difficulty level of the
blockchain. Secondly, it has additional data elements that shows that the miner
who created a block did mining on the parent blockchain and that works meet
the difficulty level of the aux blockchain. 
\newline \newline \noindent
Miners have an incentive to mine Viacoin
even if the reward is low as they can mine any other scrypt
coin with Viacoin for “free”. As Viacoin mining isn't driven by large block rewards,
this allows Viacoin to have a lower rate of inflation
compared to other cryptocurrencies that do not have support for merged mining.
\newpage

\section{Dark Gravity Wave}\label{sec: Dark Gravity Wave}
\cite{darkGravityWave}Dark Gravity Wave (DGW) is an open source difficulty algorithm. DGW was
authored by Evan Duffield, the developer and creator of X11/Darkcoin/Dash.
The algorithm was designed to address flaws like the Time warp attack in Kimoto
Gravity Wave algorithm. 
\newline \newline \noindent
Dark Gravity Wave was first introduced in Dash
(Darkcoin). DGW makes use of multiple exponential moving averages and simple
moving averages to smoothen the readjustment mechanism.
\newline Formula: \newline \newline
$2222222/ (((Difficulty+2600)/9)^2)$ \newline \newline \noindent
Dark Gravity Wave version 3 is the latest version and allows for improved difficulty retargeting compared
to the well known Kimo Gravity Well algorithm.

\section{Segwit}\label{sec: Segwit}
Viacoin has \cite{segwit}Segwit (BIP 141) activated. It helps to shrink the size of a transaction and cope with the UTXO growth.
Segregated Witness means it separates the witness from the transaction. It also
aims to increase the per-block transaction throughput by a factor of 2 or 3, while
simultaneously making block syncing faster for new nodes.
\newline \newline \noindent
The main purpose of Segwit is not to increase capacity however, it i to fix malleability and
make scripting easier to upgrade. Fixing malleability allows for other improvements in Viacoin like \cite{atomic}atomic swaps, bidirectional payment channels and
Lightning networks that could increase Viacoin interoperability with Bitcoin.
\newline \newline \noindent
Segwit includes versioning for scripts so that additional opcodes (that would normally require
a hard-fork in non-segwit transactions) can be used instead. Easier changes to script opcodes will make advancing
Viacoin easier. This makes Schnorr signatures, sidechains, MAST and other features a possibility.

\section{The Lightning Network}\label{sec: The Lightning Network}
\cite{lightningNetwork}The Lightning Network is a transfer network operating at a layer above the
Viacoin blockchain using smart contract functionality in the blockchain to enable instant payments across a network of participants. 
Enabling improvements of several orders of magnitude in
transaction throughput by moving the majority of transactions outside the
consensus ledgers into Payment channels. Allowing for millions to billions of transactions per second across the network.
A capacity that blows away legacy payment rails.
This is made possible by supporting on-chain scripts in which parties enter into bilateral stateful contracts, in which the state
can be updated by sharing a digital signature and can be closed by publishing
evidence onto the blockchain.
\newline \newline \noindent
The Lightning Network allows for exceptionally low fees. For a low-value transaction, the Lightning Network
is the silver bullet. It allows for new kinds of commerce.
By opening a payment channel with many parties, participants in the LN can
become a focal point for routing the payment of others leading into a fully
connected payment channel. The payments are enforced using a script which enforces the atomicity via decrementing time-locks.
\newline \newline \noindent
Another benefit is the possibility of atomic cross-chain transactions, enabling users to trade viacoin, bitcoin, litecoin and other Segwit coins
instantaneously, allowing for extremely efficient, decentralized exchanges or a decentralized form of 'Shapeshift.io'.
\newpage

\section{Schnorr signature}\label{sec: Schnorr signature}
We will develop Schnorr signature aggregation. It has been proposed in Bitcoin
too. It is the replacement of ECDSA for the more efficient algorithm. Until now it
was not possible to implement it without a hardfork but now it is. That’s why
Segwit was important. All sig data is moved to the witness.
Viacoin currently utilized Elliptic Curve Digital Signatures (ECDSA) as a zk proof of
ownership in order to authorize the transfer from one output to another. In 2015
Daniel J. Bernstein proposed Schnorr like signature on top of an Elliptic Curve.
\newline \newline
Some Advantages:
\begin{itemize}
\item Provably secure under standard assumptions
\item immunity to malleability
\item resistance to hash-function collisions
\item Batch validation for a 2-3x speedup
\item Native k-of-k Multisignatures \ldots
\end{itemize}
\noindent
It supports batch validation, which means if you have a group of public key
message signatures pairs rather than just a single one, you can verify if all of
them or not all of them are valid at a higher speed than each of them individually
which is exactly what we want since blocks are just big batches of signatures to
validate.
\linebreak \noindent
Native k-of-k multi signatures, the idea of Schnorr you can take multiple keys
together and have a single signature that proves that all of them are signed.
A group cat create a signature valid for the sum of keys.
U1, U2 and U3 are the users. There’s a 2 round interaction scheme where they all
come up with a nonce k1, k2, k3 and they all compute a corresponding public
point R1, R2, R3. They communicate those to each other and add them up with
an overall R value. This overall R value signs this nonce with their own key
resulting in an S1, 2, S3 and then you combine all the S values into one final S. A
signature that will be valid for the sum of their keys. This has the advantage of
the k-of-k multisig.
\linebreak

\begin{equation}
\left.
    \begin{aligned}
        U_{1} &\rightarrow k_{1},\, R_{1} \\
        U_{2} &\rightarrow k_{2},\, R_{2} \\
        U_{3} &\rightarrow k_{3},\, R_{3} \\
    \end{aligned}
\
\right|
\longrightarrow
R
\longrightarrow
\left|
\
    \begin{aligned}
        U_{1} &\rightarrow (R,\, s_{1}) \\
        U_{2} &\rightarrow (R,\, s_{2}) \\
        U_{3} &\rightarrow (R,\, s_{3}) \\
    \end{aligned}
\
\right|
\longrightarrow
(R,\, s)
\end{equation}
\newline
\noindent Even if there is not a k-of-k situation, any other policy of what combination of
keys can be signed and all one need is a merkle tree plus the ability for Schnorr
to add up and build a tree where every node leave of the tree is a combination of
keys that can be signed, hash them together and the root is the address.
OP\_CHECKSIG \& OP\_CHECKMULTISIG will be modified so that they can stack
pubkeys, delinearize and associate validated inputs and produce a
combined signature for the transaction resulting into a 20\% reduction in block
size.\newline

\noindent 2 out of 4 (k1...k4)\newline \linebreak
$O(~1)$ verification time\newline
$O(\log{}n)$ signature size\newline
$O(n)$ signing time
\newline \newline \noindent
\begin{forest}
    [Root[D[A[\textit{k1,k2}]][B[\textit{k1,k3}]]][E[C[\textit{k2,k4}][\textit{k3,k4}]]]]
\end{forest}
\newline
It is possible to do aggregation over all signatures in a single transaction. The
idea behind it is to enable system validators like Viacoin nodes to compute a
single key for every input of all transactions.
\newpage

\section{Non-atomic flushing}\label{Non-atomic flushing}
In order to make the system robust, the state on the disk has to be persistent
with a block. With an unexpected shutdown of the wallet we can startup and
rollback or roll forward inside of it and be able to get a consistent tip on the disk.
Normally whenever the cache would fill up, we would force flush
If it is present at startup it means we crashed during flush, and we rollback/roll
forward blocks inside of it to get a consistent tip on disk before proceeding.

\section{Colored coins}\label{Colored coins}
Viacoin scripting language allows to store small amounts of metadata on the
blockchain which can represent asset manipulation instructions. A Viacoin
transaction can be encoded that x units of a new asset where issued and are
credited to a Viacoin address. The term is derived from the idea of “coloring” a
nominal amount of coins.
By coloring a viacoin it turns into a token that can represent anything a user
wants to trade like a company stocks or real world value. This looks a lot like
Counterparty but there are some key differences. It uses the Viacoin blockchain
(e.g NXT).
\newline \newline \noindent
It does not issue an auxiliary coin (e.g Counterparty and Mastercoin).
The medata gives it meaning to a \cite{coloredCoins}colored coin transaction which is usually stored
in one of the OP\_RETURN opcode. The output containing the OP\_RETURN is called
a marker output. This marketcout can have a zero or non-zero value. The marker
outputs starts with the OP\_RETURN opcode and can be followed by any sequance
of opcodes which must contain a PUSHDATA opcode containing a parsable open
Asset market Payload.
The asset quantity list field is used to determine the quantity of each output of
the asset and each integer is using LEB128 encoding. If this exceeds 9 bytes, the
market output is deemed invalid. The maximum asset quantity for an output is
$2^63 - 1 units$.
The colored coins \cite{openAsset}Open Asset Protocol sits on top of the Viacoin protocol. It does
not require any changes to the Viacoin protocol.

\section{MAST (Merkelized Abstract Syntax Trees)} \label{MAST (Merkelized Abstract Syntax Trees)}
\cite{MAST}Mast allowing Viacoin transaction validation scripts to be stored in partially-hash
form and allow nodes to interact with Merkle Trees.
“When spending, users may provide only the branches they are executing, and
hashes that connect the branches to the fixed size Merkle root. This reduces the
size of redemption stack from $O(n)$ to $O(\log{}n)$ (n as the number of branches).
This enables complicated redemption conditions that is currently not possible
due to the script size and opcode limit, improves privacy by hiding unexecuted
branches, and allows inclusion of non-consensus enforced data with very low or
no additional cost.” \newline \newline \noindent
It is important because MAST allows smart contracts to be created without
clogging up the blockchain. Usually all smart contracts would be visible on the
blockchain and take up space. MAST to only reveal the smart contracts that have
been completed with saving space because nodes only read the top layer of the
Merkle Tree.
\noindent
This sounds familiar to Ethereum but there’s a difference. Ethereum access to a
VM and VIA will also obtain access to a VM though RootStock (RSK). RSK aims to
be what Ethereum is (or should have been) decentralized, Turing-complete smart
contract platform.

\section{Viacoin RSK smart contracts} \label{Viacoin RSK smart contracts}
\cite{rootstock}Rootstock is a smart contract platform which has a two-way peg. The idea is the
enable it to work with smart contracts. Rootstock runs a turing complete Virtual
machine called Rootstock Virtual Machine and is also compatible with Ethereum
virtual machine and allows solidity compiled smart contracts to run.
It could work by merge mining with Viacoin which allows the RSK blockchain to
have the same security level as Viacoin. It should allow around the 2000
transactions per second on chain and 20000 transactions per second off-chain.

\section{Anonymous transactions} \label{Anonymous transactions}
\cite{styx}An Unlinkable Anonymous Atomic Payment Hub For Viacoin based on Tumblebit.\newline
\url{https://github.com/viacoin/documents/blob/master/whitepapers/styx/Viacoin-Styx-Whitepaper.pdf}
\newpage
\printbibliography

\end{document}
